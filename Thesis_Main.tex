\documentclass[12pt,a4paper,report]{jsbook}

\usepackage{mediathesis} % 卒論用パッケージ

%%%%%%%%%%%%%%%%%%%%%%%%%%%%%%%%%%%%%%%%%%%%%%%%%%%%%%%%%%%%%%%%%%%%%%%%%%%%%%%%%%%%
%%%% タイトル・概要設定 %%%% 

% 提出年度(西暦)
\YearH{2018}

% 提出年(西暦)
\BringYear{2018}

% 提出月
\BringMonth{9}

% タイトル
\PaperTitle{リアルタイム 3DCG における}

% タイトル二行目(省略可)
\PaperTitleII{学部生と院生の温度差に関する研究}

% 著者名
\AuthorName{三次元 萌子}

% 学籍番号
\AuthorNumber{M01xxxxx}

% 指導教員名
\TeacherName{渡辺 大地 准教授}

% 指導教員名二人目(省略可)
% \TeacherNameII{阿部 雅樹 助手}

% 卒研プロジェクト名
\ProjectName{ゲームサイエンス プロジェクト}

% キーワード
\KeyWordsI{三次元、温度差、無礼講、}

% キーワード二行目(省略可、二行目を利用する場合は一行目の最後を読点で終わること)
\KeyWordsII{年齢差、軋轢}

% タイトル図ファイル
\TitleFig{./fig/titlems.eps}

% 概要
\Abstract{
\input{abstract}
}

%%%%%%%%%%%%%%%%%%%%%%%%%%%%%%%%%%%%%%%%%%%%%%%%%%%%%%%%%%%%%%%%%%%%%%%%%%%%%%%%%%%%
%%%% 寸法設定 %%%%

\setlength{\textwidth}{170truemm}		% 本文横幅
\setlength{\textheight}{230truemm}		% 本文縦幅
\setlength{\topmargin}{-5truemm}		% 上部マージン調整
\setlength{\oddsidemargin}{-4truemm}		% 奇数ページ左余白
\setlength{\evensidemargin}{\oddsidemargin}	% 偶数ページ左余白
\setlength\abovecaptionskip{-0.5truemm}		% 図キャプションと図との間隔
\setlength\belowcaptionskip{1.5truemm}		% 表キャプションと表との間隔

%%%% 各種マクロ例 %%%%

\newcommand{\bez}{\mbox{$\mathrm{B \acute{e} zier}$}}
\newcommand{\dw}[2]{\mbox{$\mathrm{{#1}_{#2}}$}}
\newcommand{\uw}[2]{\mbox{$\mathrm{{#1}^{#2}}$}}

%%%% PDF ハイパーリンク各種設定 %%%%

\hypersetup{
        bookmarksnumbered=true,
        colorlinks=false,
        setpagesize=false
}


%%%%%%%%%%%%%%%%%%%%%%%%%%%%%%%%%%%%%%%%%%%%%%%%%%%%%%%%%%%%%%%%%%%%%%%%%%%%%%%%%%%%
% ここより本文開始

\begin{document}

\pagewiselinenumbers		% 行番号付加 (本番印刷の際はコメントアウト)

\MediaThesisTitle		% タイトル作成

\pagenumbering{Roman}		% 目次用ページスタイル
\tableofcontents		% 目次
\listoffigures			% 図目次(必要に応じて有効化/無効化)
%\listoftables			% 表目次(必要に応じて有効化/無効化)
\clearpage
\pagenumbering{arabic}		% 本文用ページスタイル
\pagestyle{plain}		% 本文ページ番号位置
\baselineskip 24pt		% 行間設定

% ここから本文ファイルを挿入

\input{chap1}
\chapter{その次}
\label{chp:second}

\section{数式}
\label{sec:eqn}

数式のインラインモードは \(x^2 + y^2 \leq 1\) のように表示させることができる.
インラインモードで「\verb+$...$+」を使うやり方は,
近年の LaTeX ではあまり推奨されていないが,その利用は妨げない.

ディスプレイ数式モードを利用する際に推奨するのは equation 環境である.
\begin{equation}
	\mathbf{A}_p = \frac{\mathbf{A}\cdot\mathbf{B}}{|\mathbf{B}|^2}\mathbf{B} .
	\label{eq:samp1}
\end{equation}
数式の参照は「\verb+\ref+」ではなく「\verb+\eqref+」を用いる.
上記の数式を参照すると「式\eqref{eq:samp1}」となる.
このように,\verb+\eqref+ を用いた場合は数式中と同じ様式の括弧がつく.

また,複数行にわたる数式を表示したい場合は align 環境を用いることを推奨する.
以下の式\eqref{eq:samp2}にその例を示す.

\begin{align}
	& \begin{bmatrix}
	a_{11} & a_{12} & \cdots & a_{1n} \\
	a_{21} & a_{22} & \cdots & a_{2n} \\
	\vdots & \vdots & \ddots & \vdots \\
	a_{m1} & a_{m2} & \cdots & a_{mn} \\
	\end{bmatrix}
	\otimes
	\begin{bmatrix}
	b_{11} & b_{12} & \cdots & b_{1n} \\
	b_{21} & b_{22} & \cdots & b_{2n} \\
	\vdots & \vdots & \ddots & \vdots \\
	b_{m1} & b_{m2} & \cdots & b_{mn} \\
	\end{bmatrix} \notag \\
	& \qquad \qquad = \sum_{i}^{m}\sum_{j}^{n}a_{ij}b_{ij} .
	\label{eq:samp2}
\end{align}

eqnarray 環境は,最近の LaTeX では幾つかのパッケージと同時に利用すると
問題が発生することがあるため、利用は推奨しない.

具体的な数式の記述方法についてはWebを参照する方が手っ取り早いだろう。

\section{参考文献}
\label{sec:bib}

参考文献リストの作成は、bibtexを用いることを推奨する。
文献の参照は、リスト上で文献に付けたキーワードをciteコマンドによって指定することで記述する。
例えば、
「ポルトスら\cite{TellMePorthos}によると、
無礼講理論\cite{DayPosER_11_12}には重大な欠点が存在することが指摘されている。」
という利用方法となる。

文献を1つも参照していない状態でPDFを生成するバッチファイルを実行するとエラーとなるので注意すること。
生成したPDFファイル中で参照がうまくできていない場合には参照番号ではなく?記号が表示される。
URLの参照\cite{URLSample}の場合は、参照日も付記すること。

bibtexのリスト作成方法については、同梱してあるjxampl.bibを参考にすること。
文献の属性の種類や、設定するべきステータスについてもWebを参照すること。
論文データベースサイトではbibtexの記述形式によるテキストを出力してくれるところもあるので、
利用できると便利である。
リストの記述順は一切気にする必要がなく、
参考文献に挙げないものが含まれていても問題無いので、関連しそうな文献は全てリスト化しておくとよい。

文献リストの作成にあたってはJabRefというソフトが便利である。
Javaのランタイムをインストールする必要があるが、
文献リストをエクセルのように扱って管理することができるため、整理や分類が非常に捗る。
jxampl.bibをJabRefで開くにはJabRefのオプション、設定で規定エンコーディングをUTF8にしておく必要がある。

\section{バッチファイル}
\label{sec:bat}

今回のテンプレートには、bibtex(pbibtex) を利用することを
前提としてコンパイルを一括で行うバッチファイルを添付してある。
バッチファイルとは複数のコマンド実行をまとめて行うことが出来るもので、中身はただのテキストファイルである。

makePDF.bat をダブルクリックすればpdfファイルが生成できるようになっている。
コンパイルエラーが発生しなければコマンドプロンプト画面は勝手に終了する。
問題なくPDFが生成できた場合、意図した表示になっているかは各自で適宜確認すること。

コンパイルエラーが発生すると、エラーの説明や、
エラー箇所の行数、本文の一部が表示された状態で処理が中断する。
まずはエラーの発生箇所を確認しよう。
中断された処理は、ENTER キーでメッセージを送ることができる。
エラーの原因が確認できたら CTRL+C キーでバッチファイルの処理を終了するのが手っ取り早い。
バッチジョブを終了するか聞かれたら、Y を入力して ENTER キーで終了できる。
また、Thesis\_Main.log という名前でログファイルが生成される。

エラーを修正したあとに前回コンパイル時に生成されたファイルを一度消去しなくてはならない場合がある。
そのときには cleanFiles.bat を実行するとよい。

% \section{羽田による改変}
\label{sec:hada}

本ファイルはもともとゲームサイエンス研究室の渡辺先生が準備したものであるが,
AEDではこれをすこし修正して利用することにしている.

\subsection{改変部分について}

羽田による改変点は以下のとおりである.

\begin{enumerate}
  \item latexmkの利用\\
  バッチファイルではなく latexmk によるPDF生成の自動化を行う.そのため
  latexmkを実行するために必要なlatexmkrcファイルを追加し,マスター以外の
  ソースコードを {\tt src}ディレクトリ内におさめる形とした.
  ソースコードから作成されたファイルを消去したい場合には {\tt latexmk -C }コマンドを利用する.

  \item ソースコード名の変更と分離\\
  元となるファイルを学生番号とし,生成されるPDFファイルを学籍番号となるようにした.
  また,それ以外の{.tex}ファイルについては{\tt src }ディレクトリ内に収納することとし,
  システムが src内を検索するように設定している.
  また,パス名の設定はリスト形式なので
  複数のディレクトリからソースファイルを探したい場合には同じ形式で追加することができる.

  \item 画像ファイルの保存フォルダ\\
  画像ファイルは{\tt graphicx}パッケージを利用するが,このときにファイルを読み込むディレクトリを設定しておくことで,本文中で参照するときにディレクトリ名を変更する必要をなくしている.そのため,あとで一括して場所を変更することも可能である.
  また,パス名を設定している{\tt graphicspath }はリストで引数を取ることが可能なので,
  複数のディレクトリから画像を探したい場合には同じ形式で追加することができる.

  \item 行番号の表示
  マスターファイルの冒頭に{\tt \\pagewiselinenumbers}の行がある.
  これはページごとに小さく行番号を表示するコマンドである.もともとのファイルでは
  コメントアウトされていたがAEDではこのコマンドは最終版以外では必須のものとする.
  これによりオンラインで「XページY行目を修正して」という連絡がわかりやすくなるためである.

\end{enumerate}

\subsection{画像ファイル}
画像ファイルは 基本的に png形式ないしはjpg形式であればEPSに変換する必要はない.
ただし,画像はそのままPDFファイルに張り込まれるので必要以上に大きいサイズにしてはいけない.デジカメで撮った写真をそのままたくさん張り込むと,卒論が数百MBオーバーの大作になってしまう.(ことがある)

\subsection{githubによるファイルの管理}
卒論の作成中ファイルに関しては 中間報告と同様に {\tt github}にて管理を行うこと.
毎日作業が終わったら, commitとpush してからPCを閉じるようにしておくと,
PCが突然壊れたときなどに安心である.
ちなみに某先輩は1月になってからPCのHDDが吹っ飛んで,すべて印刷した添削原稿から
打ち込み直したという例があるので「自分は大丈夫」とは思わないこと.
			% (ファイルを作成次第有効化)
% \input{chap4}			% (ファイルを作成次第有効化)
% \input{chap5}			% (ファイルを作成次第有効化)
% \input{chap6}			% (ファイルを作成次第有効化)

\delchapnumber
\input{thanks}			% 謝辞
\bibliography{jxampl}		% BibTeXを使う場合の参考文献リスト(jxampl.bibを指定してある)
\bibliographystyle{junsrt}	% BibTeXを使う場合のスタイル
\undochapnumber
% \appendix			% 付録 (もしあれば)
% \input{appendix}		% 付録ファイルの挿入

\end{document}
