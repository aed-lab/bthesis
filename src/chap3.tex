\section{羽田による改変}
\label{sec:hada}

本ファイルはもともとゲームサイエンス研究室の渡辺先生が準備したものであるが,
AEDではこれをすこし修正して利用することにしている.

\subsection{改変部分について}

羽田による改変点は以下のとおりである.

\begin{enumerate}
  \item latexmkの利用\\
  バッチファイルではなく latexmk によるPDF生成の自動化を行う.そのため
  latexmkを実行するために必要なlatexmkrcファイルを追加し,マスター以外の
  ソースコードを {\tt src}ディレクトリ内におさめる形とした.
  ソースコードから作成されたファイルを消去したい場合には {\tt latexmk -C }コマンドを利用する.

  \item ソースコード名の変更と分離\\
  元となるファイルを学生番号とし,生成されるPDFファイルを学籍番号となるようにした.
  また,それ以外の{.tex}ファイルについては{\tt src }ディレクトリ内に収納することとし,
  システムが src内を検索するように設定している.
  また,パス名の設定はリスト形式なので
  複数のディレクトリからソースファイルを探したい場合には同じ形式で追加することができる.

  \item 画像ファイルの保存フォルダ\\
  画像ファイルは{\tt graphicx}パッケージを利用するが,このときにファイルを読み込むディレクトリを設定しておくことで,本文中で参照するときにディレクトリ名を変更する必要をなくしている.そのため,あとで一括して場所を変更することも可能である.
  また,パス名を設定している{\tt graphicspath }はリストで引数を取ることが可能なので,
  複数のディレクトリから画像を探したい場合には同じ形式で追加することができる.

  \item 行番号の表示
  マスターファイルの冒頭に{\tt \\pagewiselinenumbers}の行がある.
  これはページごとに小さく行番号を表示するコマンドである.もともとのファイルでは
  コメントアウトされていたがAEDではこのコマンドは最終版以外では必須のものとする.
  これによりオンラインで「XページY行目を修正して」という連絡がわかりやすくなるためである.

\end{enumerate}

\subsection{画像ファイル}
画像ファイルは 基本的に png形式ないしはjpg形式であればEPSに変換する必要はない.
ただし,画像はそのままPDFファイルに張り込まれるので必要以上に大きいサイズにしてはいけない.デジカメで撮った写真をそのままたくさん張り込むと,卒論が数百MBオーバーの大作になってしまう.(ことがある)

\subsection{githubによるファイルの管理}
卒論の作成中ファイルに関しては 中間報告と同様に {\tt github}にて管理を行うこと.
毎日作業が終わったら, commitとpush してからPCを閉じるようにしておくと,
PCが突然壊れたときなどに安心である.
ちなみに某先輩は1月になってからPCのHDDが吹っ飛んで,すべて印刷した添削原稿から
打ち込み直したという例があるので「自分は大丈夫」とは思わないこと.
