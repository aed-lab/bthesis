\chapter{その次}
\label{chp:second}

\section{数式}
\label{sec:eqn}

数式のインラインモードは \(x^2 + y^2 \leq 1\) のように表示させることができる.
インラインモードで「\verb+$...$+」を使うやり方は,
近年の LaTeX ではあまり推奨されていないが,その利用は妨げない.

ディスプレイ数式モードを利用する際に推奨するのは equation 環境である.
\begin{equation}
	\mathbf{A}_p = \frac{\mathbf{A}\cdot\mathbf{B}}{|\mathbf{B}|^2}\mathbf{B} .
	\label{eq:samp1}
\end{equation}
数式の参照は「\verb+\ref+」ではなく「\verb+\eqref+」を用いる.
上記の数式を参照すると「式\eqref{eq:samp1}」となる.
このように,\verb+\eqref+ を用いた場合は数式中と同じ様式の括弧がつく.

また,複数行にわたる数式を表示したい場合は align 環境を用いることを推奨する.
以下の式\eqref{eq:samp2}にその例を示す.

\begin{align}
	& \begin{bmatrix}
	a_{11} & a_{12} & \cdots & a_{1n} \\
	a_{21} & a_{22} & \cdots & a_{2n} \\
	\vdots & \vdots & \ddots & \vdots \\
	a_{m1} & a_{m2} & \cdots & a_{mn} \\
	\end{bmatrix}
	\otimes
	\begin{bmatrix}
	b_{11} & b_{12} & \cdots & b_{1n} \\
	b_{21} & b_{22} & \cdots & b_{2n} \\
	\vdots & \vdots & \ddots & \vdots \\
	b_{m1} & b_{m2} & \cdots & b_{mn} \\
	\end{bmatrix} \notag \\
	& \qquad \qquad = \sum_{i}^{m}\sum_{j}^{n}a_{ij}b_{ij} .
	\label{eq:samp2}
\end{align}

eqnarray 環境は,最近の LaTeX では幾つかのパッケージと同時に利用すると
問題が発生することがあるため、利用は推奨しない.

具体的な数式の記述方法についてはWebを参照する方が手っ取り早いだろう。

\section{参考文献}
\label{sec:bib}

参考文献リストの作成は、bibtexを用いることを推奨する。
文献の参照は、リスト上で文献に付けたキーワードをciteコマンドによって指定することで記述する。
例えば、
「ポルトスら\cite{TellMePorthos}によると、
無礼講理論\cite{DayPosER_11_12}には重大な欠点が存在することが指摘されている。」
という利用方法となる。

文献を1つも参照していない状態でPDFを生成するバッチファイルを実行するとエラーとなるので注意すること。
生成したPDFファイル中で参照がうまくできていない場合には参照番号ではなく?記号が表示される。
URLの参照\cite{URLSample}の場合は、参照日も付記すること。

bibtexのリスト作成方法については、同梱してあるjxampl.bibを参考にすること。
文献の属性の種類や、設定するべきステータスについてもWebを参照すること。
論文データベースサイトではbibtexの記述形式によるテキストを出力してくれるところもあるので、
利用できると便利である。
リストの記述順は一切気にする必要がなく、
参考文献に挙げないものが含まれていても問題無いので、関連しそうな文献は全てリスト化しておくとよい。

文献リストの作成にあたってはJabRefというソフトが便利である。
Javaのランタイムをインストールする必要があるが、
文献リストをエクセルのように扱って管理することができるため、整理や分類が非常に捗る。
jxampl.bibをJabRefで開くにはJabRefのオプション、設定で規定エンコーディングをUTF8にしておく必要がある。

\section{バッチファイル}
\label{sec:bat}

今回のテンプレートには、bibtex(pbibtex) を利用することを
前提としてコンパイルを一括で行うバッチファイルを添付してある。
バッチファイルとは複数のコマンド実行をまとめて行うことが出来るもので、中身はただのテキストファイルである。

makePDF.bat をダブルクリックすればpdfファイルが生成できるようになっている。
コンパイルエラーが発生しなければコマンドプロンプト画面は勝手に終了する。
問題なくPDFが生成できた場合、意図した表示になっているかは各自で適宜確認すること。

コンパイルエラーが発生すると、エラーの説明や、
エラー箇所の行数、本文の一部が表示された状態で処理が中断する。
まずはエラーの発生箇所を確認しよう。
中断された処理は、ENTER キーでメッセージを送ることができる。
エラーの原因が確認できたら CTRL+C キーでバッチファイルの処理を終了するのが手っ取り早い。
バッチジョブを終了するか聞かれたら、Y を入力して ENTER キーで終了できる。
また、Thesis\_Main.log という名前でログファイルが生成される。

エラーを修正したあとに前回コンパイル時に生成されたファイルを一度消去しなくてはならない場合がある。
そのときには cleanFiles.bat を実行するとよい。
